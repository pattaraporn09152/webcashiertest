\chapter{การทดสอบระบบ}
การทดสอบการทำงานขอแอนดรอยด์งแอปพลิเคชันระบบกองทุนเงินให้กู้ยืมเพื่อการศึกษา คณะวิทยาศาสตร์ มหาวิทยาลัยอุบลราชธานีและทดสอบการทำงานในส่วนของเว็บไซต์ โดยทำการทดสอบในลักษณะ Black-box Testing \cite{blackbox} หรือ Data-Driven testing ซึ่งเป็นการเทสแบบที่ไม่สนใจโปรเซส (Process) การทำงานภายในของโปรแกรมว่าทำงานอย่างไร แต่จะเน้นไปที่ Input และ Result ที่ได้มากกว่าว่าการทำงานต่าง ๆ ถูกต้องตามความต้องการ (Requirement) หรือไม่ ซึ่งการทดสอบการใช้งานแอนดรอยด์แอปพลิเคชัน และ การใช้งานเว็บแอปพลิเคชัน ได้ผลดังนี้

	\section{การทดสอบการใช้งานแอนดรอยด์แอปพลิเคชัน}
		\begin{itemize}
					\item{การทดสอบการใช้งานเมนูนำทางของแอนดรอยด์แอปพลิเคชัน}
					การทดสอบเมนูนำทางของแอปพลิเคชันในการนำทางผู้ใช้งาน ซึ่งเมนูหลักประกอบด้วย เมูนูหน้าประกาศ เมนูหน้าสนทนา เมนูหน้าปฏิทินกำหนดการ เมนูหน้าดาวน์โหลดเอกสาร เมนูส่งภาพถ่ายสำเนาเอกสาร เมนูหน้าจองคิวส่งเอกสาร เมนูหน้าคำถามทีพบบ่อย เมนูหน้าเกี่ยวกับ เมนูหน้าข้อมูลส่วนตัวและเมนูออกจากระบบ ผลทดสอบดังตารางที่ \ref{tab:ผลการทดเมนูหลัก}-\ref{tab:ผลการทดเมนูนำทาง2}
					\begin{table}[H]
						\caption{ผลการทดสอบเมนูนำทาง}
						\centering	
						\label{tab:ผลการทดเมนูหลัก}
						\begin{tabular}{ | p{4.5cm} | p{4.5cm} | p{4.5cm} | }
							\hline
							% {\setstretch{1.0} } 
							{\multicolumn{1}{c}{\centering การทำงาน}}  & 
							{\multicolumn{1}{c}{\centering เงื่อนไขการทดสอบ}} & {\multicolumn{1}{c}{\centering ผลการทดสอบ}} \\ \hline
							\setstretch{1.0}{เมนูประกาศ} 
							& \setstretch{1.0}{กดปุ่มเมนูประกาศ}
							& \setstretch{1.0}{ระบบแสดงผลหน้าจอประกาศพร้อมทั้งแสดงรายการประกาศทั้งหมด} \\ \hline
							\setstretch{1.0}{เมนูสนทนา} 
							& \setstretch{1.0}{กดปุ่มเมนูสนทนา}
							& \setstretch{1.0}{ระบบแสดงผลหน้าจอสนทนาพร้อมทั้งแสดงข้อมูลประวัติการสนทนา} \\ \cline{2-3} 
							& \setstretch{1.0}{กดปุ่มย้อนกลับ} 
							& \setstretch{1.0}{ระบบแสดงผลหน้าจอประกาศพร้อมทั้งแสดงรายการข่าวสารทั้งหมด} \\ \hline
							\setstretch{1.0}{
								เมนูหน้าปฏิทินกำหนดการ} 
							& \setstretch{1.0}{กดปุ่มเมนูปฏิทินกำหนดการ}
							& \setstretch{1.0}{ระบบแสดงผลหน้าจอสนทนาพร้อมทั้งแสดงข้อมูลประวัติการสนทนา} \\ \cline{2-3} 
							& \setstretch{1.0}{กดปุ่มย้อนกลับ} 
							& \setstretch{1.0}{ระบบแสดงผลหน้าจอประกาศพร้อมทั้งแสดงรายการข่าวสารทั้งหมด} \\ \hline
							\setstretch{1.0}{
								เมนูหน้าดาวน์โหลดเอกสาร} 
							& \setstretch{1.0}{กดปุ่มเมนูหน้าดาวน์โหลดเอกสาร}
							& \setstretch{1.0}{ระบบแสดงผลหน้าจอรายการเอกสารในระบบพร้อมทั้งแสดงปุมดาวน์โหลด} \\ \cline{2-3} 
							& \setstretch{1.0}{กดปุ่มย้อนกลับ} 
							& \setstretch{1.0}{ระบบแสดงผลหน้าจอประกาศพร้อมทั้งแสดงรายการข่าวสารทั้งหมด} \\ \hline
							\setstretch{1.0}{
								เมนูหน้าส่งภาพถ่ายสำเนาเอกสาร} 
							& \setstretch{1.0}{กดปุ่มเมนูหน้าส่งภาพสำเนาเอกสาร}
							& \setstretch{1.0}{ระบบแสดงผลหน้าส่งเอกสารภาพสำเนาเอกสาร} \\ \cline{2-3} 
							& \setstretch{1.0}{กดปุ่มย้อนกลับ} 
							& \setstretch{1.0}{ระบบแสดงผลหน้าจอประกาศพร้อมทั้งแสดงรายการข่าวสารทั้งหมด} \\ \hline
						\end{tabular}
					\end{table}
						\begin{table}[H]
							\caption{ผลการทดสอบเมนูนำทาง(ต่อ)}
							\centering	
							\label{tab:ผลการทดเมนูนำทาง2}
							\begin{tabular}{ | p{4.5cm} | p{4.5cm} | p{4.5cm} | }
								\hline
								% {\setstretch{1.0} } 
								{\multicolumn{1}{c}{\centering การทำงาน}}  & 
								{\multicolumn{1}{c}{\centering เงื่อนไขการทดสอบ}} & {\multicolumn{1}{c}{\centering ผลการทดสอบ}} \\ \hline
								\setstretch{1.0}{เมนูหน้าจองคิวส่งเอกสาร} 
								& \setstretch{1.0}{กดปุ่มเมนูหน้าจองคิวส่งเอกสาร}
								& \setstretch{1.0}{ระบบแสดงผลหน้าจองคิวส่งเอกสาร} \\ \cline{2-3} 
								& \setstretch{1.0}{กดปุ่มย้อนกลับ} 
								& \setstretch{1.0}{ระบบแสดงผลหน้าจอข่าวสารพร้อมทั้งแสดงรายการข่าวสารทั้งหมด} \\ \hline
								\setstretch{1.0}{
									เมนูหน้าคำถามทีพบบ่อย} 
								& \setstretch{1.0}{กดปุ่มเมนูหน้าคำถามทีพบบ่อย}
								& \setstretch{1.0}{ระบบแสดงหน้าจองคิวส่งเอกสาร} \\ \cline{2-3} 
								& \setstretch{1.0}{กดปุ่มย้อนกลับ} 
								& \setstretch{1.0}{ระบบแสดงผลหน้าจอประกาศพร้อมทั้งแสดงรายการข่าวสารทั้งหมด} \\ \hline
								\setstretch{1.0}{
									เมนูหน้าเกี่ยวกับ} 
								& \setstretch{1.0}{กดปุ่มเมนูหน้าเกี่ยวกับ}
								& \setstretch{1.0}{ระบบแสดงผลหน้าเกี่ยวกับซึ่งแสดงข้อมูลผู้พัฒนาวรมไปถึงแสดงเครดิต (credit) ไลบรารีต่าง ๆ ที่ใช้งานภายในแอปพลิเคชัน} \\ \cline{2-3} 
								& \setstretch{1.0}{กดปุ่มย้อนกลับ} 
								& \setstretch{1.0}{ระบบแสดงผลหน้าจอประกาศพร้อมทั้งแสดงรายการข่าวสารทั้งหมด} \\ \hline
								\setstretch{1.0}{
									เมนูหน้าบัญชีผู้ใช้} 
								& \setstretch{1.0}{กดปุ่มเมนูหน้าบัญชีผู้ใช้}
								& \setstretch{1.0}{ระบบแสดงผลหน้าจอข้อมูลส่วนตัวโดยมีข้อมูล รูปประจำตัว ชื่อผู้ใช้ สาขาวิชาและภาควิชา} \\ \cline{2-3} 
								& \setstretch{1.0}{กดปุ่มย้อนกลับ} 
								& \setstretch{1.0}{ระบบแสดงผลหน้าจอประกาศพร้อมทั้งแสดงรายการข่าวสารทั้งหมด} \\ \hline
								\setstretch{1.0}{
									เมนูออกจากระบบ} 
								& \setstretch{1.0}{กดปุ่มเมนูออกจากระบบ}
								& \setstretch{1.0}{ทำการออกจากระบบและแสดงหน้าจอข่าวสาร} \\ \hline
							\end{tabular}
						\end{table}
					
						\newpage
					\item{การทดสอบหน้ารายละเอียดประกาศ}
					ในการแสดงผลหน้าจอรายละเอียดประกาศนั้นจะประกอบไปด้วยหัวเรื่องประกาศ รายละเอียดประกาศ วันที่ประกาศและเอกสารแนบ ผลการทดสอบดังตารางที่ \ref{tab:การทดหน้ารายละเอียดประกาศ}
					\begin{table}[H]
						\caption{ผลการทดสอบหน้ารายละเอียดประกาศ}
						\centering	
						\label{tab:การทดหน้ารายละเอียดประกาศ}
						\begin{tabular}{ | p{4.5cm} | p{4.5cm} | p{4.5cm} | }
							\hline
							% {\setstretch{1.0} } 
							{\multicolumn{1}{c}{\centering การทำงาน}}  & 
							{\multicolumn{1}{c}{\centering เงื่อนไขการทดสอบ}} & {\multicolumn{1}{c}{\centering ผลการทดสอบ}} \\ \hline
							\setstretch{1.0}{หน้ารายละเอียดประกาศ} 
							& \setstretch{1.0}{กดปุ่มเมนูประกาศ}
							& \setstretch{1.0}{ระบบแสดงผลหน้าจอประกาศพร้อมทั้งแสดงรายการประกาศทั้งหมด} \\ \cline{2-3} 
							& \setstretch{1.0}{กดปุ่มอ่านรายละเอียดประกาศ} 
							& \setstretch{1.0}{ระบบแสดงผลหน้าจอรายละเอียดประกาศ} \\ \cline{2-3} 
							& \setstretch{1.0}{กดปุ่มดาวน์โหลดเอกสารแนบ} 
							& \setstretch{1.0}{ระบบแสดงผลการดาวน์โหลดเอกสารแนบ} \\ \cline{2-3} 
							& \setstretch{1.0}{เมื่อดาวน์โหลดเสร็จ กดปุ่มเปิดเอกสาร} 
							& \setstretch{1.0}{ระบบแสดงผลเอกสาร} \\ \cline{2-3} 
							& \setstretch{1.0}{กดปุ่มย้อนกลับ} 
							& \setstretch{1.0}{ระบบแสดงผลหน้าจอรายละเอียดประกาศ} \\ \cline{2-3} 
							& \setstretch{1.0}{กดปุ่มย้อนกลับอีกครั้ง} 
							& \setstretch{1.0}{ระบบแสดงผลหน้าจอประกาศพร้อมทั้งแสดงรายการข่าวสารทั้งหมด} \\ \hline
						\end{tabular}
					\end{table}
				
					\newpage
					\item{การทดสอบหน้าสนทนา}					ในการแสดงผลหน้าจอสนทนานั้นจะประกอบไปด้วยรายการประวิติการสนทนา ช่องกรอกข้อความและปุ่มส่งข้อความ  ผลการทดสอบดังตารางที่ \ref{tab:การทดสอบหน้าสนทนา}
					\begin{table}[H]
						\caption{ผลการการทดสอบหน้าสนทนา}
						\centering	
						\label{tab:การทดสอบหน้าสนทนา}
						\begin{tabular}{ | p{4.5cm} | p{4.5cm} | p{4.5cm} | }
							\hline
							% {\setstretch{1.0} } 
							{\multicolumn{1}{c}{\centering การทำงาน}}  & 
							{\multicolumn{1}{c}{\centering เงื่อนไขการทดสอบ}} & {\multicolumn{1}{c}{\centering ผลการทดสอบ}} \\ \hline
							\setstretch{1.0}{หน้ารายละเอียดประกาศ} 
							& \setstretch{1.0}{กดปุ่มเมนูสนทนา}
							& \setstretch{1.0}{ระบบแสดงผลหน้าจอสนทนาพร้อมทั้งแสดงรายการประวัติการสนทนา} \\ \cline{2-3} 
							& \setstretch{1.0}{กดปุ่มที่ช่องกรอกข้อความ} 
							& \setstretch{1.0}{ระบบแสดงตัวกระพริบ (cursor) เพื่อชี้ให้รู้ว่า ตำแหน่งของการพิมพ์อักขระ} \\ \cline{2-3} 
							& \setstretch{1.0}{พิมพ์อักขระ} 
							& \setstretch{1.0}{ระบบแสดงผลอัขระที่ถูกพิมพ์} \\ \cline{2-3} 
							& \setstretch{1.0}{กดปุ่มส่งข้อความ} 
							& \setstretch{1.0}{ระบบแสดงข้อความที่ถูกพิมพ์บนรายการประวัติสนทนาล่าสุด} \\ \cline{2-3} 
							& \setstretch{1.0}{กดปุ่มย้อนกลับ} 
							& \setstretch{1.0}{ระบบแสดงผลหน้าจอประกาศพร้อมทั้งแสดงรายการข่าวสารทั้งหมด} \\ \hline
						\end{tabular}
					\end{table}
				
					\newpage
					\item{การทดสอบหน้าปฏิทินกำหนดการ}
					ในการแสดงผลหน้าปฏิทินกำหนดการนั้นจะประกอบไปด้วยรายการประวิติการสนทนา ช่องกรอกข้อความและปุ่มส่งข้อความ ผลการทดสอบดังตารางที่ \ref{tab:การทดสอบหน้าปฏิทินกำหนดการ}
					\begin{table}[H]
						\caption{ผลการการทดสอบหน้าปฏิทินกำหนดการ}
						\centering	
						\label{tab:การทดสอบหน้าปฏิทินกำหนดการ}
						\begin{tabular}{ | p{4.5cm} | p{4.5cm} | p{4.5cm} | }
							\hline
							% {\setstretch{1.0} } 
							{\multicolumn{1}{c}{\centering การทำงาน}}  & 
							{\multicolumn{1}{c}{\centering เงื่อนไขการทดสอบ}} & {\multicolumn{1}{c}{\centering ผลการทดสอบ}} \\ \hline
							\setstretch{1.0}{หน้าปฏิทินกำหนดการ} 
							& \setstretch{1.0}{กดปุ่มเมนูปฏิทินกำหนดการ}
							& \setstretch{1.0}{ระบบแสดงหน้าจอปฏิทินกำหนดการโดยมีการแสดงกำหนดการของวันปัจจุบัน} \\ \cline{2-3} 
							& \setstretch{1.0}{กดเลือกวันที่ต้องการดูกำหนดการในปฏิทิน} 
							& \setstretch{1.0}{ระบบแสดงกำหนดการของวันที่ถูกเลือก} \\ \cline{2-3} 
							& \setstretch{1.0}{กดปุ่มย้อนกลับ} 
							& \setstretch{1.0}{ระบบแสดงผลหน้าจอประกาศพร้อมทั้งแสดงรายการข่าวสารทั้งหมด} \\ \hline
						\end{tabular}
					\end{table}
					\newpage
					
					\item{การทดสอบหน้าดาวน์โหลดเอกสาร}
					ในการแสดงผลหน้าจอดาวน์โหลดเอกสารนั้นจะประกอบไปด้วยรายการเอกสารโดยที่แต่ละฉบับบจะแสดงชื่อเอกสารและปุ่มดาวน์โหลดเอกสาร ผลการทดสอบดังตารางที่ \ref{tab:การทดสอบหน้าดาวน์โหลดเอกสาร}
					\begin{table}[H]
						\caption{ผลการการทดสอบหน้าดาวน์โหลดเอกสาร}
						\centering	
						\label{tab:การทดสอบหน้าดาวน์โหลดเอกสาร}
						\begin{tabular}{ | p{4.5cm} | p{4.5cm} | p{4.5cm} | }
							\hline
							% {\setstretch{1.0} } 
							{\multicolumn{1}{c}{\centering การทำงาน}}  & 
							{\multicolumn{1}{c}{\centering เงื่อนไขการทดสอบ}} & {\multicolumn{1}{c}{\centering ผลการทดสอบ}} \\ \hline
							\setstretch{1.0}{หน้าดาวน์โหลดเอกสาร} 
							& \setstretch{1.0}{กดปุ่มเมนูหน้าดาวน์โหลดเอกสาร}
							& \setstretch{1.0}{ระบบแสดงผลหน้าจอรายการเอกสารในระบบพร้อมทั้งแสดงปุมดาวน์โหลด} \\ \cline{2-3} 
							& \setstretch{1.0}{กดปุ่มดาวน์โหลดเอกสาร} 
							& \setstretch{1.0}{ระบบแสดงผลการดาวน์โหลดเอกสาร} \\ \cline{2-3} 
							& \setstretch{1.0}{เมื่อดาวน์โหลดเสร็จ กดปุ่มเปิดเอกสาร} 
							& \setstretch{1.0}{ระบบแสดงผลเอกสาร} \\ \cline{2-3} 
							& \setstretch{1.0}{กดปุ่มย้อนกลับ} 
							& \setstretch{1.0}{ระบบแสดงผลหน้าจอรายการเอกสารในระบบพร้อมทั้งแสดงปุมดาวน์โหลด} \\ \cline{2-3} 
							& \setstretch{1.0}{กดปุ่มย้อนกลับ} 
							& \setstretch{1.0}{ระบบแสดงผลหน้าจอประกาศพร้อมทั้งแสดงรายการข่าวสารทั้งหมด} \\ \hline
						\end{tabular}
					\end{table}
				
					\newpage
					\item{การทดสอบหน้าส่งภาพถ่ายสำเนาเอกสาร}
%					\item การทดสอบหน้าส่งภาพถ่ายสำเนาเอกสาร\\
					ในการแสดงผลหน้าส่งภาพถ่ายสำเนาเอกสารนั้นจะประกอบไปด้วยปุ่มเพิ่มเอกสารฉบับที่ 1 ปุ่มเพิ่มเอกสารฉบับที่ 2 และปุ่มส่งเอกสาร ผลการทดสอบดังตารางที่ \ref{tab:การทดสอบหน้าส่งภาพถ่ายสำเนาเอกสาร}
					\begin{table}[H]
						\caption{ผลการทดสอบหน้าส่งภาพถ่ายสำเนาเอกสาร}
						\centering	
						\label{tab:การทดสอบหน้าส่งภาพถ่ายสำเนาเอกสาร}
						\begin{tabular}{ | p{4.5cm} | p{4.5cm} | p{4.5cm} | }
							\hline
							% {\setstretch{1.0} } 
							{\multicolumn{1}{c}{\centering การทำงาน}}  & 
							{\multicolumn{1}{c}{\centering เงื่อนไขการทดสอบ}} & {\multicolumn{1}{c}{\centering ผลการทดสอบ}} \\ \hline
							\setstretch{1.0}{หน้าส่งภาพถ่ายสำเนาเอกสาร} 
							& \setstretch{1.0}{กดปุ่มเมนูหน้าจองคิวส่งเอกสาร}
							& \setstretch{1.0}{ระบบแสดงหน้าจองคิวส่งเอกสาร} \\ \cline{2-3} 
							& \setstretch{1.0}{กดปุ่มเพิ่มเอกสารฉบับที่ 1} 
							& \setstretch{1.0}{ระบบแสดงหน้าจอกล้องถ่ายภาพ} \\ \cline{2-3} 
							& \setstretch{1.0}{กดปุ่มถ่ายภาพเอกสาร} 
							& \setstretch{1.0}{ระบบแสดงผลภาพเอกสาร} \\ \cline{2-3} 
							& \setstretch{1.0}{กดปุ่มถัดไป} 
							& \setstretch{1.0}{ระบบแสดงผลหน้าปรัปแต่งภาพเอกสาร} \\ \cline{2-3} 
							& \setstretch{1.0}{กดปุ่มยืนยัน} 
							& \setstretch{1.0}{ระบบแสดงผลภาพหน้าส่งภาพถ่ายสำเนาเอกสารพร้อมทั้งแสดงผลภาพเอกสารฉบับที่ 1} \\ \cline{2-3} 
							& \setstretch{1.0}{กดปุ่มเพิ่มเอกสารฉบับที่ 2} 
							& \setstretch{1.0}{ระบบแสดงหน้าจอกล้องถ่ายภาพ} \\ \cline{2-3} 
							& \setstretch{1.0}{กดปุ่มถ่ายภาพเอกสาร} 
							& \setstretch{1.0}{ระบบแสดงผลภาพเอกสาร} \\ \cline{2-3} 
							& \setstretch{1.0}{กดปุ่มถัดไป} 
							& \setstretch{1.0}{ระบบแสดงผลหน้าปรัปแต่งภาพเอกสาร} \\ \cline{2-3} 
							& \setstretch{1.0}{กดปุ่มยืนยัน} 
							& \setstretch{1.0}{ระบบแสดงผลภาพหน้าส่งภาพถ่ายสำเนาเอกสารพร้อมทั้งแสดงผลภาพเอกสารฉบับที่ 1 และฉบับที่ 2} \\ \cline{2-3}
							& \setstretch{1.0}{กดส่งเอกสาร} 
							& \setstretch{1.0}{ระบบแสดงผลการส่งเอกสารและแสดงสถานะการตรวจเอกสาร}  \\ \cline{2-3}
							& \setstretch{1.0}{กดปุ่มย้อนกลับ} 
							& \setstretch{1.0}{ระบบแสดงผลหน้าจอประกาศพร้อมทั้งแสดงรายการข่าวสารทั้งหมด} \\ \hline
						\end{tabular}
					\end{table}
					\newpage
					
					\item{การทดสอบหน้าจองคิวส่งเอกสาร}
					ในการแสดงผลหน้าจองคิวส่งเอกสารนั้นจะประกอบไปด้วยปุ่มกดเลือกวันที่ ปุ่มกดเลือเวลา ผลการทดสอบดังตารางที่ \ref{tab:การทดสอบหน้าจองคิวส่งเอกสาร}
					\begin{table}[H]
						\caption{ผลการทดสอบหน้าจองคิวส่งเอกสาร}
						\centering	
						\label{tab:การทดสอบหน้าจองคิวส่งเอกสาร}
						\begin{tabular}{ | p{4.5cm} | p{4.5cm} | p{4.5cm} | }
							\hline
							% {\setstretch{1.0} } 
							{\multicolumn{1}{c}{\centering การทำงาน}}  & 
							{\multicolumn{1}{c}{\centering เงื่อนไขการทดสอบ}} & {\multicolumn{1}{c}{\centering ผลการทดสอบ}} \\ \hline
							\setstretch{1.0}{หน้าจองคิวส่งเอกสาร} 
							& \setstretch{1.0}{กดปุ่มเมนูหน้าจองคิวส่งเอกสาร}
							& \setstretch{1.0}{ระบบแสดงผลหน้าจอกำหนดการส่งเอกสาร} \\ \cline{2-3} 
							& \setstretch{1.0}{กดปุ่มเลือกวันที่ต้องการส่งเอกสาร} 
							& \setstretch{1.0}{ระบบแสดงผลวันที่ถูกเลือก} \\ \cline{2-3} 
							& \setstretch{1.0}{กดปุ่มเลือกเวลาที่ต้องการส่งเอกสาร} 
							& \setstretch{1.0}{ระบบแสดงผลเวลาที่ถูกเลือกพร้อมทั้งแสดงปุ่มกดบันทึก} \\ \cline{2-3} 
							& \setstretch{1.0}{กดปุ่มบันทึก} 
							& \setstretch{1.0}{ระบบแสดงผลการจองวันที่ส่งเอกสาร} \\ \cline{2-3}
							& \setstretch{1.0}{กดปุ่มย้อนกลับ} 
							& \setstretch{1.0}{ระบบแสดงผลหน้าจอประกาศพร้อมทั้งแสดงรายการข่าวสารทั้งหมด} \\ \hline
						\end{tabular}
					\end{table}
				\end{itemize}
%	\end{enumerate}

\section{การทดสอบการใช้งานเว็บแอปพลิเคชัน}
	\begin{itemize}
	\item{การทดสอบการใช้งานเมนูนำทางของเว็บแอปพลิเคชัน}
	การทดสอบเมนูนำทางของเว็บแอปพลิเคชันในการนำทางผู้ใช้งาน ซึ่งเมนูหลักประกอบด้วย เมูนูหน้าประกาศ เมนูหน้าสนทนา เมนูหน้าปฏิทินกำหนดการ เมนูหน้าดาวน์โหลดเอกสาร เมนูหน้าคำถามทีพบบ่อย เมนูหน้าเกี่ยวกับและเมนูออกจากระบบ ผลทดสอบดังตารางที่ \ref{tab:ผลการทดเมนูหลัก}-\ref{tab:ผลการทดเมนูนำทาง2}
	\begin{table}[H]
		\caption{ผลการทดสอบเมนูนำทาง}
		\centering	
		\label{tab:ผลการทดเมนูหลัก}
		\begin{tabular}{ | p{4.5cm} | p{4.5cm} | p{4.5cm} | }
			\hline
			% {\setstretch{1.0} } 
			{\multicolumn{1}{c}{\centering การทำงาน}}  & 
			{\multicolumn{1}{c}{\centering เงื่อนไขการทดสอบ}} & {\multicolumn{1}{c}{\centering ผลการทดสอบ}} \\ \hline
			\setstretch{1.0}{เมนูประกาศ} 
			& \setstretch{1.0}{กดปุ่มเมนูประกาศ}
			& \setstretch{1.0}{ระบบแสดงผลหน้าจอประกาศพร้อมทั้งแสดงรายการประกาศทั้งหมด} \\ \hline
			\setstretch{1.0}{เมนูสนทนา} 
			& \setstretch{1.0}{กดปุ่มเมนูสนทนา}
			& \setstretch{1.0}{ระบบแสดงผลหน้าจอสนทนาพร้อมทั้งแสดงข้อมูลประวัติการสนทนา} \\ \cline{2-3} 
			& \setstretch{1.0}{กดปุ่มย้อนกลับ} 
			& \setstretch{1.0}{ระบบแสดงผลหน้าจอประกาศพร้อมทั้งแสดงรายการข่าวสารทั้งหมด} \\ \hline
			\setstretch{1.0}{
				เมนูหน้าปฏิทินกำหนดการ} 
			& \setstretch{1.0}{กดปุ่มเมนูปฏิทินกำหนดการ}
			& \setstretch{1.0}{ระบบแสดงผลหน้าจอสนทนาพร้อมทั้งแสดงข้อมูลประวัติการสนทนา} \\ \cline{2-3} 
			& \setstretch{1.0}{กดปุ่มย้อนกลับ} 
			& \setstretch{1.0}{ระบบแสดงผลหน้าจอประกาศพร้อมทั้งแสดงรายการข่าวสารทั้งหมด} \\ \hline
			\setstretch{1.0}{
				เมนูหน้าดาวน์โหลดเอกสาร} 
			& \setstretch{1.0}{กดปุ่มเมนูหน้าดาวน์โหลดเอกสาร}
			& \setstretch{1.0}{ระบบแสดงผลหน้าจอตารางรายการเอกสารในระบบพร้อมทั้งแสดงปุมดาวน์โหลด} \\ \cline{2-3} 
			& \setstretch{1.0}{กดปุ่มย้อนกลับ} 
			& \setstretch{1.0}{ระบบแสดงผลหน้าจอประกาศพร้อมทั้งแสดงรายการข่าวสารทั้งหมด} \\ \hline
			\setstretch{1.0}{
			เมนูหน้าคำถามทีพบบ่อย} 
		& \setstretch{1.0}{กดปุ่มเมนูหน้าคำถามทีพบบ่อย}
		& \setstretch{1.0}{ระบบแสดงหน้าจองคิวส่งเอกสาร} \\ \cline{2-3} 
		& \setstretch{1.0}{กดปุ่มย้อนกลับ} 
		& \setstretch{1.0}{ระบบแสดงผลหน้าจอประกาศพร้อมทั้งแสดงรายการข่าวสารทั้งหมด} \\ \hline
			\setstretch{1.0}{
				เมนูหน้าเกี่ยวกับ} 
			& \setstretch{1.0}{กดปุ่มเมนูหน้าเกี่ยวกับ}
			& \setstretch{1.0}{ระบบแสดงผลหน้าเกี่ยวกับซึ่งแสดงข้อมูลผู้พัฒนาวรมไปถึงแสดงเครดิต (credit) ไลบรารีต่าง ๆ ที่ใช้งานภายในแอปพลิเคชัน} \\ \cline{2-3} 
			& \setstretch{1.0}{กดปุ่มย้อนกลับ} 
			& \setstretch{1.0}{ระบบแสดงผลหน้าจอประกาศพร้อมทั้งแสดงรายการข่าวสารทั้งหมด} \\ \hline
			\setstretch{1.0}{
				เมนูออกจากระบบ} 
			& \setstretch{1.0}{กดปุ่มเมนูออกจากระบบ}
			& \setstretch{1.0}{ทำการออกจากระบบและแสดงหน้าจอข่าวสาร} \\ \hline
		\end{tabular}
	\end{table}
	
	\newpage
	\item{การทดสอบหน้ารายละเอียดประกาศ}
	ในการแสดงผลหน้าจอรายละเอียดประกาศนั้นจะประกอบไปด้วยหัวเรื่องประกาศ รายละเอียดประกาศ วันที่ประกาศและเอกสารแนบ ผลการทดสอบดังตารางที่ \ref{tab:การทดหน้ารายละเอียดประกาศ}
	\begin{table}[H]
		\caption{ผลการทดสอบหน้ารายละเอียดประกาศ}
		\centering	
		\label{tab:การทดหน้ารายละเอียดประกาศ}
		\begin{tabular}{ | p{4.5cm} | p{4.5cm} | p{4.5cm} | }
			\hline
			% {\setstretch{1.0} } 
			{\multicolumn{1}{c}{\centering การทำงาน}}  & 
			{\multicolumn{1}{c}{\centering เงื่อนไขการทดสอบ}} & {\multicolumn{1}{c}{\centering ผลการทดสอบ}} \\ \hline
			\setstretch{1.0}{หน้ารายละเอียดประกาศ} 
			& \setstretch{1.0}{กดปุ่มเมนูประกาศ}
			& \setstretch{1.0}{ระบบแสดงผลหน้าจอประกาศพร้อมทั้งแสดงรายการประกาศทั้งหมด} \\ \cline{2-3} 
			& \setstretch{1.0}{กดปุ่มอ่านรายละเอียดประกาศ} 
			& \setstretch{1.0}{ระบบแสดงผลหน้าจอรายละเอียดประกาศ} \\ \cline{2-3} 
			& \setstretch{1.0}{กดปุ่มดาวน์โหลดเอกสารแนบ} 
			& \setstretch{1.0}{ระบบแสดงผลการดาวน์โหลดเอกสารแนบ} \\ \cline{2-3} 
			& \setstretch{1.0}{กดปุ่มย้อนกลับ} 
			& \setstretch{1.0}{ระบบแสดงผลหน้าจอประกาศพร้อมทั้งแสดงรายการข่าวสารทั้งหมด} \\ \hline
		\end{tabular}
	\end{table}
	
	\newpage
	\item{การทดสอบหน้าสนทนา}					ในการแสดงผลหน้าจอสนทนานั้นจะประกอบไปด้วยรายการประวิติการสนทนา ช่องกรอกข้อความและปุ่มส่งข้อความ  ผลการทดสอบดังตารางที่ \ref{tab:การทดสอบหน้าสนทนา}
	\begin{table}[H]
		\caption{ผลการการทดสอบหน้าสนทนา}
		\centering	
		\label{tab:การทดสอบหน้าสนทนา}
		\begin{tabular}{ | p{4.5cm} | p{4.5cm} | p{4.5cm} | }
			\hline
			% {\setstretch{1.0} } 
			{\multicolumn{1}{c}{\centering การทำงาน}}  & 
			{\multicolumn{1}{c}{\centering เงื่อนไขการทดสอบ}} & {\multicolumn{1}{c}{\centering ผลการทดสอบ}} \\ \hline
			\setstretch{1.0}{หน้ารายละเอียดประกาศ} 
			& \setstretch{1.0}{กดปุ่มเมนูสนทนา}
			& \setstretch{1.0}{ระบบแสดงผลหน้าจอสนทนาพร้อมทั้งแสดงรายการประวัติการสนทนา} \\ \cline{2-3} 
			& \setstretch{1.0}{กดปุ่มที่ช่องกรอกข้อความ} 
			& \setstretch{1.0}{ระบบแสดงตัวกระพริบ (cursor) เพื่อชี้ให้รู้ว่า ตำแหน่งของการพิมพ์อักขระ} \\ \cline{2-3} 
			& \setstretch{1.0}{พิมพ์อักขระ} 
			& \setstretch{1.0}{ระบบแสดงผลอัขระที่ถูกพิมพ์} \\ \cline{2-3} 
			& \setstretch{1.0}{กดปุ่มส่งข้อความ} 
			& \setstretch{1.0}{ระบบแสดงข้อความที่ถูกพิมพ์บนรายการประวัติสนทนาล่าสุด} \\ \cline{2-3} 
			& \setstretch{1.0}{กดปุ่มย้อนกลับ} 
			& \setstretch{1.0}{ระบบแสดงผลหน้าจอประกาศพร้อมทั้งแสดงรายการข่าวสารทั้งหมด} \\ \hline
		\end{tabular}
	\end{table}
	
	\newpage
	\item{การทดสอบหน้าปฏิทินกำหนดการ}
	ในการแสดงผลหน้าปฏิทินกำหนดการนั้นจะประกอบไปด้วยรายการประวิติการสนทนา ช่องกรอกข้อความและปุ่มส่งข้อความ ผลการทดสอบดังตารางที่ \ref{tab:การทดสอบหน้าปฏิทินกำหนดการ}
	\begin{table}[H]
		\caption{ผลการการทดสอบหน้าปฏิทินกำหนดการ}
		\centering	
		\label{tab:การทดสอบหน้าปฏิทินกำหนดการ}
		\begin{tabular}{ | p{4.5cm} | p{4.5cm} | p{4.5cm} | }
			\hline
			% {\setstretch{1.0} } 
			{\multicolumn{1}{c}{\centering การทำงาน}}  & 
			{\multicolumn{1}{c}{\centering เงื่อนไขการทดสอบ}} & {\multicolumn{1}{c}{\centering ผลการทดสอบ}} \\ \hline
			\setstretch{1.0}{หน้าปฏิทินกำหนดการ} 
			& \setstretch{1.0}{กดปุ่มเมนูปฏิทินกำหนดการ}
			& \setstretch{1.0}{ระบบแสดงตารางกำหนดารทั้งหมด} \\ \cline{2-3} 
			& \setstretch{1.0}{กดปุ่มย้อนกลับ} 
			& \setstretch{1.0}{ระบบแสดงผลหน้าจอประกาศพร้อมทั้งแสดงรายการข่าวสารทั้งหมด} \\ \hline
		\end{tabular}
	\end{table}
	\newpage
	
	\item{การทดสอบหน้าดาวน์โหลดเอกสาร}
	ในการแสดงผลหน้าจอดาวน์โหลดเอกสารนั้นจะประกอบไปด้วยรายการเอกสารโดยที่แต่ละฉบับบจะแสดงชื่อเอกสารและปุ่มดาวน์โหลดเอกสาร ผลการทดสอบดังตารางที่ \ref{tab:การทดสอบหน้าดาวน์โหลดเอกสาร}
	\begin{table}[H]
		\caption{ผลการการทดสอบหน้าดาวน์โหลดเอกสาร}
		\centering	
		\label{tab:การทดสอบหน้าดาวน์โหลดเอกสาร}
		\begin{tabular}{ | p{4.5cm} | p{4.5cm} | p{4.5cm} | }
			\hline
			% {\setstretch{1.0} } 
			{\multicolumn{1}{c}{\centering การทำงาน}}  & 
			{\multicolumn{1}{c}{\centering เงื่อนไขการทดสอบ}} & {\multicolumn{1}{c}{\centering ผลการทดสอบ}} \\ \hline
			\setstretch{1.0}{หน้าดาวน์โหลดเอกสาร} 
			& \setstretch{1.0}{กดปุ่มเมนูหน้าดาวน์โหลดเอกสาร}
			& \setstretch{1.0}{ระบบแสดงผลหน้าจอตารางรายการเอกสารในระบบพร้อมทั้งแสดงปุมดาวน์โหลด} \\ \cline{2-3} 
			& \setstretch{1.0}{กดปุ่มดาวน์โหลดเอกสาร} 
			& \setstretch{1.0}{ระบบแสดงผลการดาวน์โหลดเอกสาร} \\ \cline{2-3} 
			& \setstretch{1.0}{กดปุ่มย้อนกลับ} 
			& \setstretch{1.0}{ระบบแสดงผลหน้าจอรายการเอกสารในระบบพร้อมทั้งแสดงปุมดาวน์โหลด} \\ \cline{2-3} 
			& \setstretch{1.0}{กดปุ่มย้อนกลับ} 
			& \setstretch{1.0}{ระบบแสดงผลหน้าจอประกาศพร้อมทั้งแสดงรายการข่าวสารทั้งหมด} \\ \hline
		\end{tabular}
	\end{table}
\end{itemize}
