\chapter{บทนำ}

\section{ที่มาและเหตุผล }
   ปัจจุบันเทคโนโลยีเกี่ยวกับการประมวลผลภาพได้ถูกนำไปประยุกต์ใช้มากมาย เช่น การสแกนใบหน้าเพื่อเข้าใช้งานโทรศัพท์มือถือ การสเเกนลายนิ้วมือ อีกทั้งการนำไปจดจำใบหน้าเพื่อเก็บข้อมูล เเละนำข้อมูลมาใช้ในครั้งต่อไป  และเทคโนโลยีการประมวลผลภาพนี้ยังเป็นที่นิยมในสังคมตอนนี้ เพราะมันง่ายต่อการใช้งาน ไม่ต้องมีอุปกรณ์ที่ช่วยในการเข้าถึงข้อมูล ใช้เเค่ใบหน้าในการตรวจสอบ  อีกทั้งยังมีการนำ Machine learning มาใช้ในการช่วยตัดสินใจในการสั่งสินค้าของลูกค้า  



ด้วยเหตุนี้จึงมีความประสงค์ที่จะนำเทคโนโลยีการประมวลผลภาพมาใช้ เพื่อที่จะบันทึกใบหน้าของลูกค้าที่มาชำระเงิน เพื่อนำข้อมูลมาใช้ในครั้งต่อๆไป แทนการใช้อุปกรณ์ที่ช่วยในการเข้าถึงข้อมูลลูกค้าที่ยุ่งยาก เเละสะดวกต่อลูกค้าเพื่อช่วยแก้ไขการลืมบัตรสมาชิก สะกดชื่อไม่ถูก ลืมเบอร์โทรศัพท์ที่ใช้ในการสมัคร  และช่วยในการตัดสินใจสั่งสินค้าได้รวดเร็วขึ้น โดยใช้ Decision Tree มาช่วยในการเเนะนำเมนู 


\section{วัตถุประสงค์}
\begin{enumerate}
	\item เพื่อออกแบบเเละพัฒนาแอปพลิเคชั่น จดจำใบหน้าในการชำระเงิน 
	\item  เพื่อดูข้อมูล และประวัติการสั่งซื้อที่ผ่านมาของลูกค้าได้ โดยผ่านการสแกนใบหน้าลูกค้า    
	\item เพื่อแนะนำ เมนูแก่ลูกค้า เวลาที่ลูกค้าไม่รู้ว่าจะสั่งอะไร 
	\item ใช้การสแกนใบหน้าแทนการใช้บัตรสมาชิก 
    \item ช่วยในการตัดสินใจ การสั่งซื้อสินค้าของลูกค้าได้     
    \item เพื่อช่วยลดทรัพยากรที่ใช้ในการผลิตบัตรสมาชิก และอปุกรณ์อื่นๆที่ต้องใช้ร่วมกัน
\end{enumerate}
\section{ขอบเขตของโครงงาน}
\begin{enumerate}
	\item ใช้สำหรับพนักงานเเคชเชียร์เท่านั้น
	\item เพื่อดูข้อมูล และประวัติการสั่งซื้อที่ผ่านมาของลูกค้าได้อย่างรวดเร็ว โดยผ่านการสแกนใบหน้าลูกค้าเท่านั้น
	\item ไม่ต้องมีบัตร ก็สามารถดูข้อมูลต่างๆของลูกค้าได้ จากการสแกนใบหน้า
    \item 	ช่วยในการตัดสินใจการสั่งซื้อสินค้าของลูกค้าได้ 
    \item พนักงานเเคชเชียร์สามารถสั่งสินค้าให้ลูกค้าได้เท่านั้น
		 \begin{itemize}
		 	
		 	
	
	\end{itemize}
\end{enumerate}
\section{ประโยชน์ที่คาดว่าจะได้รับ}
\begin{enumerate}
	\item ช่วยลดปริมาณทรัพยากรที่ใช้ในการผลิตบัตร 
	\item ช่วยให้พนักงานเเคชเชียร์ทำงานได้ถูกต้องและเเม่นยำ 
   \item ช่วยลดปัญหาลูกค้าลืมบัตรสมาชิก ลืมชื่อที่ใช้สมัครสมาชิก
   \item สามารถแนะนำเมนูให้ลูกค้าโดยเทียบจากประวัติการสั่งซื้อ 
\end{enumerate}
\section{เครื่องมือที่ใช้ในการพัฒนา (Development tools)}
\subsection{ฮาร์ดเเวร์}
\begin{enumerate}
	\item สมาร์ทโฟน (Smart phone)
		\begin{itemize}
			\item ทำงานบนระบบปฏิบัติการแอนดรอย์เวอร์ชัน 5.0 หรือ API Level 21
			\item หน่วยประมาลผลกลาง Mediatek MT6753 Octa-core ความเร็ว 1.3 กิกะเฮิร์ตซ์ (Gigahertz, GHz)
			\item หน่วยประมวลผลกราฟฟิกอย่างน้อย Mali-T720MP3
			\item หน่วยความจำหลักอย่างน้อย 2 กิกะไบต์ (Gigabyte, GB)
			\item หน่วยความจำสำรองอย่างน้อย 16 กิกะไบต์ (Gigabyte, GB)
			\item หน้าจอแสดงผลความละเอียดอย่างน้อย 1080 x 1920 พิกเซล  (Pixel)
			\item หน้าจอแสดงผลขนาดอย่างน้อย 5 นิ้ว
			\item กล้องถ่ายรูปความละเอียดอย่างน้อย 13 เมกกะพิกเซล (Magapixel)
		\end{itemize}
	
	\item เครื่องคอมพิวเตอร์ส่วนบุคคล (Personal computer)
		\begin{itemize}
			\item  ทำงานบนระบบปฏิบัติการ Elementary os พื้นฐานการทำงานบน Linux
			\item  หน่วยประมวลผลกลาง Intel Core i3-3217U ความเร็ว 1.80 กิกะเฮิร์ตซ์ (Gigahertz, GHz)
			\item  หน่วยประมวลผลกราฟฟิก NVIDIA GeForce GT 720M ความจำ 2 กิกะไบต์ (Gigabyte, GB) 
			\item  หน่วยความจำหลัก 4 กิกะไบต์ (Gigabyte, GB)
			\item  หน่วยความจำสำรอง 120 กิกะไบต์ (Gigabyte, GB)
		\end{itemize}
\end{enumerate}

\subsection{ซอฟต์แวร์ (Software)}
\begin{enumerate}
	\item Visaul Studio Code เป็นโปรแกรมสำหรับพัฒนาเว็บเเอปพลเคชั่น พัฒนาด้วยภาษา HTML, JavaScript, Python เเละ Java 
	\item Django เป็น Framework ที่ใช้สำหรับสร้าง UI สำหรับระบบ หรือเรียกอีกอย่างว่า Backend Framework ใช้สำหรับเป็นเว็บเซิฟเวอร์ (Web Server)  
	\item OpenCv เป็นไลบรารีฟังก์ชันการเขียนโปรแกรมที่มุ่งเน้นไปทาง การเเสดงผลแบบเรียลไทม์
	ซึ่งนำมาใช้ในส่วนของการสแกนใบหน้า
	\item Machine Learning ในส่วนของการเรียนรู้ของอัลกอริทึม Association Rule	
	\item Sqlite3 เป็นตัวจัดการฐานข้อมูล

\end{enumerate}

\newpage
\subsection{แผนการดำเนินการ}
	ในการสร้างระบบสแกนใบหน้าสำหรับเเคชเชียร์ ผู้พัฒนาได้แบ่งขั้นตอนการดำเนินงานไว้ด้วยกัน 7 ขั้นตอน ดังต่อไปนี้

%\begin{landscape}
%\sffamily
\begin{table}[H]
	\noindent
	\caption{ขั้นตอนการดำเนินงาน}
	\begin{ganttchart}[
		canvas/.append style={fill=none, draw=black!5, line width=.75pt},
		vgrid={*2{draw=black!7, line width=.75pt}},
		title label font=\bfseries\footnotesize,
		bar label node/.append style={
			align=left,
			text width=width("7. Functional Testing On")},
		bar/.append style={draw=none, fill=black!63}
		]{1}{22}
		\gantttitle{2562}{3}
		\gantttitle{2563}{5}\\
		\gantttitle{ต.ค.}{2}
		\gantttitle{พ.ย.}{2}
		\gantttitle{ธ.ค.}{2} 
		\gantttitle{ม.ค.}{2}
		\gantttitle{ก.พ.}{2}
		\gantttitle{มี.ค.}{2}
		\gantttitle{เม.ย.}{2}
		\gantttitle{พ.ค.}{2}
		 \\
		\ganttbar{1.ศึกษาความเป็นไปได้}{1}{4} \\
		\ganttbar{2.เสนอหัวข้อโครงงาน}{5}{6} \\
		\ganttbar{3.ศึกษาค้นคว้าข้อมูล}{7}{8} \\
		\ganttbar{4.ศึกษาการใช้เครื่องมือ}{7}{10} \\
		\ganttbar{5.วิเคราะห์และออกแบบ}{7}{12} \\
		\ganttbar{6.เขียนโปรแกรม}{7}{13} \\
		\ganttbar{7.ทดสอบและแก้ปัญหา}{13}{14} \\
		\ganttbar{8.จัดทำเอกสาร}{13}{16} \\
	\end{ganttchart}
	\label{tab:ganttchart}
\end{table}
%\end{landscape}
%TODO แก้เทมเพลตเอาชื่อตารางไว้ด้านบน
